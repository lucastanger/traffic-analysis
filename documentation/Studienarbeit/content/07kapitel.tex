%!TEX root = ../dokumentation.tex
\chapter{Abschluss}

Im folgenden wird der Abschluss dieser Studienarbeit dargestellt. Es wurden die theoretischen Grundlagen und der Stand der Technik zum Zeitpunkt der Erstellung untersucht. Hierbei wurden spezifische Vorgehensweisen bei maschinellen Lernverfahren erläutert. Mit den daraus hervorgehenden Erkenntnissen wurde ein neuronales Netz entwickelt und trainiert, sowie ein Prototyp zur Erkennung von Verkehrsaufkommen erstellt.  

\section{Fazit}

Die Arbeit behandelte die Recherche neuronaler Netze und maschineller Lernverfahren in Zusammenhang mit der Erkennung von Objekten im Straßenverkehr. Im Detail wurde hierbei auf die Funktionsweise der neuronalen Netze eingegangen und Grundlagen bezüglich Tensoren, \acp{ANN} und \acp{CNN} dargelegt. 

Das in Kapitel \ref{sec:model} entwickelte Modell umfasst theoretisches Wissen in einer praxisnahen Implementierung. Das Modell wurde als beispielhafte Implementierung einer \gls{mc_classification} entworfen. Zunächst wurde hierfür die Vorverarbeitung der Daten erläutert, welche dem bekannten Problem des Overfittings entgegenwirkt. Die Problematik des Overfittings wurde in Kapitel \ref{sec:finalmodel} dargestellt und in den zugehörigen Schaubildern visualisiert. Durch den Einsatz von Augmentierungsverfahren wie der zufälligen Ausrichtung, Farbanpassung, Rotation und Inversion der Bildtensoren wurde der Trainingsdatensatz erweitert und somit eine effektive Gegenmaßnahme für die zuvor erwähnte Problematik eingeführt. Der anschließende Entwurf eines Netzwerks zur Objektklassifikation wurde in mehreren Evaluierungsschritten iterativ verbessert. Hierfür wurde ein initiales Modell entworfen, welches daraufhin mit Verlust- und Genauigkeitsmetriken evaluiert wurde. Da die anfängliche Implementierung keine befriedigenden Ergebnisse erzeugen konnte, wurden Anpassungen an den Hyperparametern getroffen. Weitergehend wurde die Verwendung verschiedener Architekturen getestet. Hierbei sind die Anzahl an Convolutional Layern, Größe der Feature-Map und des Fully-Connected Layers sowie die Ermittlung des Dropout Schwellenwertes analysiert worden. Die Definition des finalen Modells kombinierte alle gewonnenen Erkenntnisse. Ein Vergleich zwischen dem ursprünglichen und dem endgültigen Modell rechtfertigt die getroffene Entscheidung.

Eine prototypische Implementierung, welche die gewünschten Anforderungen umsetzt, wurde in Kapitel \ref{sec:prototypical_implementation} eingeführt. Dabei wurde unter anderem der Prozess der Bewegungserkennung mit Hilfe der OpenCV Hintergrundsubtraktion ermöglicht. Die Vorgehensweise ist anhand gängiger Algorithmen erläutert und unter Verwendung etablierter Literatur angeführt worden. Die schrittweise Entwicklung der Bewegungserkennung ist mit Hilfe von Codeausschnitten und Beispielbildern visualisiert worden. Darauf aufbauend wurde die Objekterkennung, unter Verwendung der Tensorflow Object Detection \ac{API}, umgesetzt. Die Selektion des vordefinierten EfficientDet-D7 Modells wurde ausführlich begründet. Die Realisierung der Objekterkennung erfolgte visuell begleitet durch mehrere exemplarische Ausschnitte. Um Geschwindigkeit und Fahrtrichtung der Objekte zu ermitteln, wurde der \ac{SORT} Algorithmus verwendet. Um ein Grundwissen über den verwendeten Algorithmus zu vermitteln, sind die Eigenschaften und verwendeten Techniken vorgestellt und in den Prototyp übernommen worden. Die Messung der Geschwindigkeit erfolgte abschließend durch Verwendung einer eigens definierten Formel. Die Prototypische Implementierung schloss mit der Visualiserung und Erhebung von Statistiken ab. 

\section{Ausblick}

Das Ergebnis dieser Studienarbeit ermöglicht das Erkennen von Objekten im Straßenverkehr mit der Erhebung von zugehörigen Statistiken, wie der Anzahl von erkannten Objekten und deren Geschwindigkeiten. Mit dem Entwurf eines neuronalen Netzes zur Klassifikation von Objekten wäre es lohnenswert, den darin enthaltenen Multi-Class Anwendungsfall in einen Multi-Label Anwendungsfall zu transformieren, um somit die Performanz des implementierten Prototypen anhand eines eigenen Modells evaluieren zu können. Begleitend hierzu ist ein Datensatz notwendig, welcher den verwendeten \ac{CIFAR}-10 Datensatz in seiner Größe und Vielfalt der Objekte übertrifft. 

Eine Optimierung des Prototypen kann mit Hilfe des auf dem \ac{SORT} Algorithmus basierenden Trackingalgorithmus DeepSORT erreicht werden \cite{DBLP:journals/corr/WojkeBP17}. Hierbei können Identitätswechsel in der Objekterkennung, durch den verbesserten Umgang mit verdeckten Objekten, reduziert werden. Diesbezüglich wird automatisch eine Verbesserung der Fahrtrichtungs- und Geschwindigkeitserkennung erwirkt. Die Performanz des Prototypen kann darüber hinaus durch eine gezieltere Objekterkennung ermöglicht werden. Hierfür muss die Objekterkennung auf den erkannten Bereich der Bewegungserkennung angewandt werden und nicht auf dem Gesamtbild des Videos. 

Anhand der vorgestellten Ergebnisse ließe sich zudem herausfinden, ob eine Erweiterung des Funktionsumfangs die Software in Zukunft für komplexere Anwendungsbereiche zugänglich machen würde. 