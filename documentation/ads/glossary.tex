%!TEX root = ../dokumentation.tex

%
% vorher in Konsole folgendes aufrufen:
%	makeglossaries makeglossaries dokumentation.acn && makeglossaries dokumentation.glo
%

%
% Glossareintraege --> referenz, name, beschreibung
% Aufruf mit \gls{...}
%
\newglossaryentry{Glossareintrag}{name={Glossareintrag},plural={Glossareinträge},description={Ein Glossar beschreibt verschiedenste Dinge in kurzen Worten}}

\newglossaryentry{Kernel}
	{name={Kernel}, 
	description={Ein Programm, das den Code des Anwenders ausführt und introspektiert}
	}
	
\newglossaryentry{Adam}
	{name={Adam},
	description = {Erweiterung des RMSProp Optimizers}
	}
	
\newglossaryentry{AlexNet}
	{name={AlexNet}, 
	description={Ein von Alexander Krizhevsky entworfenes \ac{CNN}}
	}

\newglossaryentry{Bias}
	{name={Bias}, 
	description={Unabhängiges Gewicht eines neuronalen Netzes}
	}
	
\newglossaryentry{Hyperparameter}
	{name={Hyperparameter},
	description={Wert zur Steuerung des Lernprozesses}
	}

\newglossaryentry{Bin}
	{
	name={Bin},
	plural={Bins},
	description={Klasse in einem Histogram mit konstanter oder variabler Breite}
	}
	
\newglossaryentry{BoundingBox}
	{
	name={Bounding Box},
	description={Begrenzungsrahmen für die Formkompatibilität}
	}
	
\newglossaryentry{mc_classification}
	{
	name={Multi-Class Classification},
	description={Klassifizierungsmodell zur Erkennung einzelner Klassen in einem Bild}
	}
	
\newglossaryentry{ml_classification}
	{
	name={Multi-Label Classification},
	description={Klassifizierungsmodell zur Erkennung multipler Labels in einem Bild}
	}
	
\newglossaryentry{Numpy}
	{
	name = {Numpy},
	description = {Programmierbibliothek in Python für die einfache Handhabung von Vektoren, Matrizen und mehrdimensionalen Arrays}
	}