%!TEX root = ../dokumentation.tex
\chapter{Abschluss}
Im Folgenden wird der Abschluss dieser Arbeit dargestellt. Die Vorteile einer eigenständig entworfenen Software gegenüber der marktetablierten Ignite Lösung wurden dargelegt, dabei wurde ein Vorschlag für die konzeptionelle Umsetzung unter Berücksichtigung aller Anforderungen erstellt. Weitergehend wird ein Fazit aus den gewonnenen Erfahrungen bezüglich der Konzeption entnommen.
\section{Fazit}
Aus Kapitel \ref{sec:concept} geht hervor, dass das erarbeitete Konzept technisch möglich und durchführbar ist. Um das Konzept als erfolgreich werten zu können, muss eine weitergehende Evaluierung statt finden, welche die prozessbeteiligten Personen mit einbezieht. Im Rahmen der Konzeption wurde auf die aus der Umfrage erhobenen Bedenken eingegangen. Komplexeren Bedenken wurden Lösungsansätze entgegengebracht, die jedoch dem Zeitpunkt der Implementierung und dessen Stand der Technik angepasst werden müssen. Hierbei ist zu erwähnen, dass die dargestellten Umsetzungsvorschläge lediglich Varianten offenlegen und bei einer evidenten Umsetzung gegebenenfalls modifiziert werden müssen.

Um dem Funktionsumfang von etablierten Crowdfunding Lösungen, wie zum Beispiel Kickstarter und Indiegogo, entsprechen zu können, sind weitere Untersuchungen notwendig. Einerseits muss abgewägt werden, ob bei der Implementierung einer eigenen Crowdfunding Lösung der Aspekt der Kernkompetenzen von camos nicht überschätzt wird und ob somit eine Implementierung überhaupt wirtschaftlich tragbar ist. Der Versuch, eine eigene und unabhängige Crowdfunding Lösung zu bieten, welche Mitarbeiter und Geschäftsleitung gleichermaßen zufriedenstellt, geht mit Schwierigkeiten einher. So müssen auf der einen Seite die gestellten Anforderungen berücksichtigt werden, auf der anderen Seite darf der Standpunkt der gegenwärtigen Entscheidungsgremien nicht gänzlich ignoriert werden. Die Umsetzung dieser Anforderungen bedarf einiger Zeit in der Implementierung und Pflege. In Anbetracht dessen, muss die Frage beantwortet werden, ob eine externe Lösung unter Modifikation des eigenen Prozesses die bessere Lösung wäre. Die Entwicklung dieser externen Crowdfunding Lösungen erfährt meist den vollen Fokus der jeweiligen Anbieter und dementsprechend der Entwicklungskapazitäten. Hieraus resultiert, dass eine Abwägung von Aufwand und Nutzen vorgenommen werden muss. 

\section{Ausblick}

Durch die in dieser Arbeit gestellte Umfrage und des daraus hervorgehenden Konzepts ist die Akzeptanz einer Crowdfunding Lösung analysiert. Es wäre darüber hinaus lohnenswert, das Konzept zu evaluieren. Hierfür kann den Mitgliedern der Entscheidungsgremien, sowie den Geschäftsleitern, ein Evaluierungsbogen zugesandt werden, indem das erhobene Konzept auf seine Vollständigkeit geprüft wird. Die daraus gewonnenen Erfahrungen können, in näherer Zusammenarbeit mit der Geschäftsleitung, in eine überarbeitete Version des Konzepts einfließen. 

Wie schon zuvor in Kapitel \ref{sec:concept} erwähnt, ist es für die erfolgreiche Implementierung einer Crowdfunding Plattform elementar, der Mitarbeitercrowd den Nutzen hinter dieser Plattform darzulegen. Zur Aufklärung kann ein dafür vorgesehenes Dokument angefertigt werden, welches die Modifikation des Prozesses transparent darstellt und für jeden Mitarbeiter zu jeder Zeit verfügbar ist. Über den Umfang dieser Arbeit hinaus wird das Konzept intern weiterentwickelt, um in naher Zukunft einen ersten Prototypen verwenden zu können.
