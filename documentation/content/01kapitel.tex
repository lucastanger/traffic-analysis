%!TEX root = ../dokumentation.tex

\chapter{Einleitung}

In der heutigen Zeit werden Algorithmen immer häufiger eingesetzt, um verschiedenen Personengruppen die Auswertung von Daten leichter zu machen. Maschinelle Lernverfahren bieten mit der Zeit immer fortgeschrittenere Möglichkeiten, unterschiedlichste Alltagssituationen zu analysieren. Im Zusammenhang mit dem Thema Verkehrsanalyse bietet das maschinelle Lernen vielfältige Möglichkeiten, den öffentlichen Raum zu verbessern. Hauptbestandteil dieser Arbeit soll es sein, eine Verkehrsstatistik unter Verwendung eines Klassifikationsalgorithmus zu erstellen. Anleitend hierzu wird ein eigenes Neuronales Netz trainiert, um dem Leser die Grundlagen des Themas näher zu bringen. Die daraus gewonnenen Erkenntnisse werden in dieser Arbeit wiedergegeben. Aus den resultierenden Ergebnissen wird eine einfache Ableitung über das Verkehrsaufkommen erreicht. 

\section{Motivation und Problemstellung}



\section{Zielsetzung}



\section{Methodik und Aufbau der Arbeit}


