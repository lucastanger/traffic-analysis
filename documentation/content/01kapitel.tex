%!TEX root = ../dokumentation.tex

\chapter{Einleitung}

In der heutigen Zeit werden Algorithmen immer häufiger eingesetzt, um verschiedenen Personengruppen die Auswertung von Daten leichter zu machen. Maschinelle Lernverfahren bieten mit der Zeit immer fortgeschrittenere Möglichkeiten, unterschiedlichste Alltagssituationen zu analysieren. Im Zusammenhang mit dem Thema Verkehrsanalyse bietet das maschinelle Lernen vielfältige Möglichkeiten, den öffentlichen Raum zu verbessern. Hauptbestandteil dieser Arbeit soll es sein, eine Verkehrsstatistik unter Verwendung eines Klassifikationsalgorithmus zu erstellen. Anleitend hierzu wird ein eigenes Neuronales Netz trainiert, um dem Leser die Grundlagen des Themas näher zu bringen. Die daraus gewonnenen Erkenntnisse werden in dieser Arbeit wiedergegeben. Aus den resultierenden Ergebnissen wird eine einfache Ableitung über das Verkehrsaufkommen erreicht. 

In den letzten Jahren wurde die Fahrzeugklassifizierung als Teil des breiteren Forschungsgebiets der Fahrzeugerkennung umfassend untersucht. Ein Fahrzeugklassifizierungssystem ist unerlässlich für effektive Transportsysteme (z. B. Verkehrsmanagement und Mautsysteme), Parkoptimierung, Strafverfolgung und autonome Navigation. Ein gängiger Ansatz verwendet bildverarbeitungsbasierte Methoden und nutzt externe physikalische Merkmale, um ein Fahrzeug in Standbildern und Videoströmen zu erkennen und zu klassifizieren. Ein Mensch ist zwar in der Lage, die Klasse eines Fahrzeugs mit einem kurzen Blick auf die digitalen Daten (Bild, Video) zu identifizieren, jedoch ist dies mit einem Computer nicht so einfach zu bewerkstelligen. Verschiedene Probleme wie Verdeckung, Verfolgung eines sich bewegenden Objekts, Schatten, Rotation, fehlende Farbinvarianz und vieles mehr müssen sorgfältig berücksichtigt werden, um ein effektives und robustes automatisches Fahrzeugklassifizierungssystem zu entwickeln, das unter realen Bedingungen arbeiten kann. Diese Arbeit soll den aktuellen Stand der Technik erforschen, ein Modell zur maschinengestützten Erkennung von Objekten erzeugen sowie eine prototypische Implementierung vorstellen, mit Hilfe welcher die Erkennung und Verfolgung von Objekten im Straßenverkehr möglich ist. 
